% !TEX program = xelatex

\documentclass{resume}
\usepackage{changepage}
%\usepackage{zh_CN-Adobefonts_external} % Simplified Chinese Support using external fonts (./fonts/zh_CN-Adobe/)
%\usepackage{zh_CN-Adobefonts_internal} % Simplified Chinese Support using system fonts

\begin{document}
\pagenumbering{gobble} % suppress displaying page number

\name{Sahil Mokashi}

\basicInfo{
  \email{sahilmokashi6.1@gmail.com} \textperiodcentered\ 
  \phone{(+91) 8971767006} \textperiodcentered\ 
  \linkedin[sahil-mokashi]{https://www.linkedin.com/in/sahil-mokashi}}

\section{\faGraduationCap\ Profile Summary}
\textit{Data Engineer} with a strong background in Google Cloud technologies and Python, with a track record of designing, optimizing data pipelines and leveraging Google Cloud services like Google Cloud Storage, BigQuery, Pub/Sub, Dataflow. 
\begin{itemize}
  \item <2 year experience working in Data Engineering
  \item Proficient with working on Cloud Technologies.
  \item Proficient in Python, SQL and worked with handling data, visualization, and exploratory data analysis.
  \item Experienced with working on PySpark (GCP - Dataproc) for data processing and ETL.
  \item Experience with working on most public cloud technologies (GCP / AZURE / AWS) with expertise in GCP.
  \item 2x Google Cloud certified.
  \item 1x Azure Cloud certified.
\end{itemize}


\section{\faGraduationCap\ Education}
\datedsubsection{\textbf{S.G. Balekundri Institute of Technology}, Belgaum , India}{2018 -- 2022}
\textit{Bachelors} in Computer Science and Engineering (CSE) \hfill cgpa - 8.05

\section{\faUsers\ Experience}

\datedsubsection{\textbf{Persistent Systems}, Pune, India}{2022 -- Present}

\begin{adjustwidth}{0.75cm}{0pt} % Indentation of 1cm
\subsubsection*{\textbf{Software Engineer}}{Jul. 2022 -- Present}
\role{Data Engineering - Python, GCP, PySpark, SQL, ETL}
  \item \textbf{POC : Amazon Reviews Analysis(GCP)} - 
     This project scrapes reviews from amazon for specific options using tools with selenium , Data is processed, analysed and Data is stored in Bigquery with reports. Python-Flask based app that is used to trigger the features.
\begin{itemize}
  \item Worked on creating pipelines using Apache Beam (DataFlow) for data ingestion to GCS.
  \item Used cloud functions as triggers for the processing jobs.
  \item Used PySpark (DataProc) for data processing from GCS bucket.
  \item Loaded the data into data warehouse (Bigquery) with analysis reports using Dataflow pipelines.
  \item Worked with airflow dags to schedule the jobs.

\\
\end{itemize}
 \item \textbf{WingMate} - 
     This is a GENAI tool that combines use of muliple generative AI via the use of API's , this adds features like code translations from python to ruby , testcase generation ,database gen , Git automation ,etc. Built on Python , Django and leverages cloud services(AWS).
\begin{itemize}
  \item Worked on data processing and ETL using PySpark .
  \item worked with creating a docker image for the app
  \item worked with github actions to automate the process of creating the Docker image and deploying the image to ECR repository , and deploying it to AWS Apprunner.
\end{itemize}
\\
 \item \textbf{ETL} - 
     ETL tasks on cloud using PYTHON.
\begin{itemize}
  \item Trained and worked Used beam pipelines , spark jobs  , airflow
  \item tasks revolving around data engineering(ETL) over cloud
  \item worked with Python and Sql extensively
  \item worked on data migrations 
  \item use of Dataflow, cloud Storage, Dataproc, PySpark composer to accomplish tasks in optimised and efficient way

\end{itemize}


\subsubsection*{\textbf{Software Engineering Intern}}{Feb. 2022 -- Jul. 2022}
\role{Python, Java, Spring/SpringBoot}
\item  
\begin{itemize}
  \item worked using core java
  \item Worked on creating microservices using technologies spring / springboot 
  \item worked with mysql database, along with rest api's.
\end{itemize}


\end{adjustwidth}

\datedsubsection{\textbf{Uilatech}, Belgaum, India}{Sept. 2021 -- Oct. 2021}
\subsubsection*{\textbf{Intern}}
\role{Python, Image Processing, OpenCV}
\textit{Academic internship trained and worked on various machine learning concepts focusing on OpenCv and Image Processing }
\begin{itemize}
  \item worked on multiple mini projects for image processing and OpenCv. 
  \item used pretrained models for for various general purpose classifications.
  \item used mediapipe for movement detection in varios ML projects. 
\end{itemize}

% Reference Test
%\datedsubsection{\textbf{Paper Title\cite{zaharia2012resilient}}}{May. 2015}
%An xxx optimized for xxx\cite{verma2015large}
%\begin{itemize}
%  \item main contribution
%\end{itemize}

\section{\faCogs\ Skills}
\begin{itemize}[parsep=0.5ex]
  \item Programming Languages: Python , Java
  \item Data: PySpark , SQL , Apache Beam , Apache Airflow ,ETL
  \item Platform: windows
  \item Cloud : GCP (proficient) ,Azure and AWS 
  \item Operations: GIT
  \item other skills beginner: Flask , Django, Rest, spring/springboot.
\end{itemize}

\section{\faHeartO\ Honors and Awards}
\datedline{\textit{{WINNER} }of Persistent AI Technathon}{Jul. 2023}
\textit{Secured first place in the AI Hackathon , Built a java/spring based microservice that serves as a common notification platform for mail/sms/push notification. Use of inbuilt java mail sender and AWS (Ses,Sns,Push Notification) services. Successfully able to send mail and sms notifications using Rest API \\ }
\datedline{Top performer of Interns batch 2022 at Persistent
}{Jul. 2022}
\textit{Awarded for being among the top performers of Interns batch
2022(GEMS FY23) at Persistent Systems.\\}

\datedline{\textit{Winner} of Project Exhibition at S.G.B.I.T 
}{Mar. 2019}
\textit{Secured 1st place at project exhibition held in my college(2nd year) for "IOT based Pollution monitoring system". Arduino Uno based Pollution
(smoke) monitoring system that uses smoke detector to detect
pollution.It sends the readings to 'thingspeak' which can be accessed via
internet. This system triggers a sound alarm at the device and the user can monitor the pollution levels via their phone/web.}

\section{\faGraduationCap\ Certifications}
\begin{itemize}[parsep=0.5ex]
  \item \datedline{Google Could Certified : Associate Cloud Engineer}
{Oct. 2022}
  \item \datedline{Google Could Certified : Professional Data Engineer}
{Mar. 2023}
  \item \datedline{Microsoft : Azure Fundamentals AZ-900}
{Jul. 2022}
\end{itemize}



\section{\faInfo\ Miscellaneous}
\begin{itemize}[parsep=0.5ex]
  \item GitHub: https://github.com/Sahil-A-Mokashi
  \item Languages: English - Fluent, Hindi - Fluent , Kannada - Fluent , Marathi - limited working proficiency
  \item Leetcode: https://leetcode.com/sahil-mokashi/
\end{itemize}

%% Reference
%\newpage
%\bibliographystyle{IEEETran}
%\bibliography{mycite}
\end{document}
